\documentclass[11pt, letterpaper]{article}

% Encoding
\usepackage[T1]{fontenc} % Output
\usepackage[latin1]{inputenc} % Input

% MATHS
\usepackage{amsmath, amssymb, amsfonts} % Math stuff
% linenofix -------------------------------------------
\newcommand*\patchAmsMathEnvironmentForLineno[1]{%
  \expandafter\let\csname old#1\expandafter\endcsname\csname #1\endcsname
  \expandafter\let\csname oldend#1\expandafter\endcsname\csname end#1\endcsname
  \renewenvironment{#1}%
     {\linenomath\csname old#1\endcsname}%
     {\csname oldend#1\endcsname\endlinenomath}}% 
\newcommand*\patchBothAmsMathEnvironmentsForLineno[1]{%
  \patchAmsMathEnvironmentForLineno{#1}%
  \patchAmsMathEnvironmentForLineno{#1*}}%
\AtBeginDocument{%
\patchBothAmsMathEnvironmentsForLineno{equation}%
\patchBothAmsMathEnvironmentsForLineno{align}%
\patchBothAmsMathEnvironmentsForLineno{flalign}%
\patchBothAmsMathEnvironmentsForLineno{alignat}%
\patchBothAmsMathEnvironmentsForLineno{gather}%
\patchBothAmsMathEnvironmentsForLineno{multline}%
}
%-----------------------------------------------------

\usepackage{mhsetup, mathtools, empheq} % Other optional maths stuff
\usepackage{mathrsfs}
\usepackage{amsthm}
\newenvironment{myproof}{\begin{proof}[\unskip\nopunct]}{\end{proof}}

% FORMATTING
\usepackage{array} % Tables etc
\usepackage{longtable} % Break table over pages
\usepackage{pdflscape} % to turn table in landscape mode

\usepackage{multirow} % Join cells over multiple rows
\usepackage{paralist} % In-paragraph itemized lists (\begin{inparaenum}\item.... \end{inparaenum})
\usepackage{lineno} % For line numbering
\usepackage{setspace} % Line spacing
\usepackage{nameref} % Reference to section name, not just number

\usepackage{chngcntr} % For equation counter in the appendix

% HEAD AND FOOT
\usepackage{lastpage}
\usepackage{fancyhdr}
\fancypagestyle{maintext}{
\fancyhf{}
\fancyhead[C]{}
\fancyfoot[C]{\thepage/\pageref{LastPage}}
\renewcommand{\headrulewidth}{0.pt}
}

\fancypagestyle{appendix}{
\fancyhf{}
\fancyhead[C]{}
\fancyfoot[C]{\thepage/\pageref{LastPage}}
\renewcommand{\headrulewidth}{0.pt}
}

% BIBLIOGRAPHY
\usepackage[]{natbib} % Package for the bibliography

% Colors
\usepackage[table]{xcolor}

% FIGURES
% Package to label the different panels of a figure
\usepackage[FIGTOPCAP, raggedright, sf, bf, scriptsize, SF, nooneline]{subfigure} 
% Tikz and related packages (to draw figures)
\usepackage{tikz}
\usetikzlibrary{positioning,shadows,arrows,trees, shapes, fit, calc}
% Style of the captions
\usepackage[font={small,sf, doublespacing}, labelfont=bf]{caption}


\usepackage{floatrow}

%% FONTS
% Fonts of the main text
\usepackage{fourier}
% Helvetica for sans serif fonts
\renewcommand{\sfdefault}{phv}
% Make sure mathcal looks the same
\DeclareMathAlphabet{\mathcal}{OMS}{cmsy}{m}{n}

% SECTIONS
\usepackage{titlesec}
\usepackage{needspace}

%% Section
\titleformat{\section}[block]
  {\needspace{2in}\Large \bfseries\sffamily}
  {\thesection}
  {1em}
  {}
% Subsection
\titleformat{\subsection}[block]
  {\large\bfseries\sffamily}
  {\thesubsection}
  {1em}
  {} 
% Subsubsection
\titleformat{\subsubsection}[block]
  {\bfseries\sffamily}
  {\thesubsubsection}
  {1em}
  {} 
% Paragraph
\titleformat{\paragraph}[runin]
  {\sffamily\normalsize\bfseries}
  {\theparagraph}
  {0em}
  {}

% TABLES
\usepackage{multirow}
\usepackage{rotating}

\usepackage[colorlinks=true, citecolor=black, linkcolor=black, urlcolor=black]{hyperref}
\hypersetup{
%    pdftitle={Mutation and social evolution},
 %   pdfauthor={Your name here},
 %   pdfsubject={Your subject here},
  %  pdfkeywords={keyword1, keyword2},
  %  bookmarksnumbered=true,     
    bookmarksopen=true,         
  %  bookmarksopenlevel=1,       
  %  colorlinks=true,            
    pdfstartview=Fit,           
    pdfpagemode=UseOutlines,    % this is the option you were lookin for
  %  pdfpagelayout=TwoPageRight
}


% STYLES IMPORTED FOR EQREF DOCUMENT
\makeatletter 
\renewcommand{\eqref}[1]{\textup{{\normalfont eq.~(\ref{#1}}\normalfont)}}
\makeatother
\newcommand{\Eqref}[1]{Eq.~(\ref{#1})}
\newcommand{\sysref}[1]{system~(\ref{#1})}
\newcommand{\Sysref}[2]{System~(\ref{#1})}

\usepackage{mathrsfs}
\usepackage{dsfont} % Mathbb for 1


%%% STUFF TO COMMENT OUT FOR THE FINAL VERSION 
\usepackage{soul} % Highlighting (command \hl{})
\usepackage{todonotes} % Add Word-style comments (\todo{})
\usepackage[]{showlabels} % To display the labels of the equations on the pdf; 
\usepackage[top=0.75in, bottom = 0.75in, left = 1.5in, right = 1.5in]{geometry}
%\geometry{textwidth=0.65\paperwidth}
%%%

\newcommand{\ie}{\textit{i.\!e.\!}}
\newcommand{\eg}{\textit{e.\!g.\!}}

\newcommand{\deriv}[2]{\partial_{#2}\!{#1}\,}
%\newcommand{\deriv}[2]{\left.\frac{\partial #1}{\partial #2}\right |_{#2=0}}
\newcommand{\derivv}[3]{\left.\frac{\partial #1}{\partial #2}\right |_{#3=0}} % Version of \deriv with different evaluation
\newcommand{\derivvv}[3]{\left.\frac{\partial #1}{\partial #2}\right |_{#3}} % Version of \deriv with different evaluation
\newcommand{\sderivv}[3]{\partial_{#2}\!{#1}\,} % Version of \deriv with different evaluation, shorter

% Expectations and Probabilities
\newcommand{\Esp}[1]{\mathbb{E}\big[ #1\big]}%{\mathbb{E}\left[ #1\right]}
\newcommand{\Espzero}[1]{\mathbb{E}_0\big[ #1\big]}%{\mathbb{E}\left[ #1\right]}
\newcommand{\Espt}[2]{\mathbb{E}_{#2}\big[ #1\big]}%{\mathbb{E}\left[ #1\right]}
\newcommand{\Prob}[1]{\mathbb{P}\left[ #1\right]}
\newcommand{\Probzero}[1]{\mathbb{P}_0\left[ #1\right]}
 
\newcommand{\bigO}[1]{O\left( #1 \right)}
\newcommand{\Tr}[1]{\textrm{Tr}\left( #1\right)}

\newcommand{\subst}[1]{\scriptsize \substack{ #1}}
\newcommand{\justif}[1]{\quad \textrm{\small [#1]}}

\newcommand{\ssum}[2]{{\textstyle \sum\limits_{#1}^{#2}}}

\newcommand{\appname}[0]{Appendix}

% NOTATION FOR QUELLER TYPE MATRIX
\newcommand{\bb}{\mathsf{b}}
\newcommand{\cc}{\mathsf{c}}
\newcommand{\dd}{\mathsf{d}}

\newcommand{\direct}{\mathrm{D}}
\newcommand{\indirect}{\mathrm{I}}

\newcommand{\makemat}[1]{\mathbf{#1}}
\newcommand{\Pin}{P_{\textrm{in}}}
\newcommand{\Pout}{P_{\textrm{out}}}

\newcommand{\Moran}{\textrm{M}}
\newcommand{\BD}{\textrm{BD}}
\newcommand{\DB}{\textrm{DB}}
\newcommand{\WF}{\textrm{WF}}
\makeatletter
\let\thetitle\@title
\let\theauthor\@author
\makeatother

\usepackage{hyphenat}

\newcommand{\ein}{e_{\textrm{in}}}
\newcommand{\eself}{e_{\textrm{self}}}
\newcommand{\eout}{e_{\textrm{out}}}
\newcommand{\din}{d_{\textrm{in}}}
\newcommand{\dself}{d_{\textrm{self}}}
\newcommand{\dout}{d_{\textrm{out}}}
\newcommand{\Qin}{Q_{\textrm{in}}}
\newcommand{\Qout}{Q_{\textrm{out}}}

\newcommand{\ndemes}{N_D}
% FLUSH FIGURES IN THE END
\usepackage[tablesonly]{endfloat}

%\usepackage[fighead, nofiglist, nomarkers]{endfloat}
\renewcommand{\efloatseparator}{}

\newcommand{\defeq}{\mathrel{\mathop:}=}

\begin{document}
\section*{Moran DB life-cycle, sums unpacked}
\pagestyle{appendix}
The table below contains all combinations for $i$, $j$, $k$, $l$ involved in the four sums. \\
\textbf{Notation:} $(i, j)$ means that $i$ and $j$ are in the same deme, but are different; $G_i$ refers to the deme containing site $i$.

%\begin{landscape}

{\scriptsize
\rowcolors{2}{gray!25}{white}
\begin{longtable}{>{$}l<{$} >{$}l<{$} >{$}l<{$} >{$}l<{$}   >{$}l<{$}   >{$}l<{$}   >{$}l<{$}   >{$}l<{$}  >{$}l<{$} >{$}l<{$} }
\rowcolor{gray!60}
%\hline
& j & k & l & \textrm{Notation} & \textrm{Count}  & d_{ji} & d_{li} & e_{kl} & Q_{jk} \\
%\hline
\endhead % all the lines above this will be repeated on every page
%
1 & j=i & k=i & l = i & (i = j= k =l) & 1 & \dself & \dself & \eself & 1 \\*
%
2 & j=i & k=i & l\neq i; l \in G_i & (i=j=k, l) & n-1 & \dself & \din & \ein & 1\\*
%
3 & j=i & k=i & l\not \in G_i & (i=j=k), (l) & N-n & \dself & \dout & \eout & 1 \\
%
%
4 & j=i & k\neq i; k\in G_i & l=i & (i=j=l, k) & n-1 & \dself & \dself & \ein & \Qin \\*
%
5 & j=i & k\neq i; k\in G_i & l=k & (i=j, k=l) & n-1 & \dself & \din & \eself & \Qin \\*
%
6 & j=i & k\neq i; k\in G_i & l\neq i, k; l\in G_i & (i=j, k, l) & (n-1)(n-2) & \dself & \din & \ein & \Qin \\*
%
7 & j=i & k\neq i; k\in G_i & l\not \in G_i & (i=j, k), (l) & (n-1)(N-n) & \dself &  \dout & \eout & \Qin \\
%
%
8 & j=i & k\not \in G_i & l=i=j & (i=j=l), (k) & (N-n) & \dself & \dself & \eout & \Qout \\*
%
9 & j=i & k\not \in G_i & l\neq i, l\in G_i & (i=j, l), (k) & (N-n)(n-1) & \dself & \din & \eout & \Qout \\*
%
10 & j=i & k\not \in G_i & l=k & (i=j), (k=l) & (N-n) & \dself & \dout & \eself & \Qout \\*
%
11 & j=i & k\not \in G_i & l\neq k; l\in G_k & (i=j), (k, l) & (N-n)(n-1) & \dself & \dout & \ein & \Qout \\*
%
12 & j=i & k\not \in G_i & l\not \in G_i, G_k & (i=j), (k), (l) & (N-n)(N-2n) & \dself & \dout & \eout & \Qout \\
%
%
%
%
13 & j\neq i, j\in G_i & k=i & l=i & (i=k=l, j) & (n-1) & \din & \dself & \eself & \Qin \\*
%
14 & j\neq i, j\in G_i & k=i & l=j & (i=k, j=l) & (n-1) & \din & \din & \ein & \Qin \\*
%
15 & j\neq i, j\in G_i & k=i & l\neq i, j; l\in G_i & (i=k, j, l) & (n-1)(n-2) & \din & \din & \ein & \Qin \\*
%
16 & j\neq i, j\in G_i & k=i & l\not \in G_i & (i=k, j), (l) & (n-1)(N-n) & \din & \dout & \eout & \Qin \\
%
%
17 & j\neq i, j\in G_i & k=j & l=i & (i=l, j=k) & (n-1) & \din & \dself & \ein & 1 \\*
%
18 & j\neq i, j\in G_i & k=j & l=j & (i, j=k=l) & (n-1) & \din &  \din & \eself & 1\\*
%
19 & j\neq i, j\in G_i & k=j & l\neq i, j; l\in G_i & (i, j=k, l) & (n-1)(n-2) & \din & \din & \ein & 1\\*
%
20 & j\neq i, j\in G_i & k=j & l\not \in G_i & (i, j=k), (l)& (n-1)(N-n) & \din & \dout & \eout & 1\\
%
%
21 & j\neq i, j\in G_i & k\neq i,j; k\in G_i & l=i & (i=l, j, k) & (n-1)(n-2) & \din & \dself & \ein & \Qin \\*
%
22 & j\neq i, j\in G_i & k\neq i,j; k\in G_i & l=j & (i,j=l,k) & (n-1)(n-2) & \din & \din & \ein & \Qin \\*
%
23 & j\neq i, j\in G_i & k\neq i,j; k\in G_i & l=k & (i,j,k=l) & (n-1)(n-2) & \din & \din & \eself & \Qin  \\*
%
24 & j\neq i, j\in G_i & k\neq i,j; k\in G_i & l\neq i,j,k; l
\in G_i & (i,j,k,l) & (n-1)(n-2)(n-3) & \din & \din & \ein & \Qin \\*
%
25 & j\neq i, j\in G_i & k\neq i,j; k\in G_i & l\not \in G_i & (i, j, k), (l) & (n-1)(n-2)(N-n)  & \din & \dout & \eout & \Qin \\
%
%
26 & j\neq i; j\in G_i & k\not\in G_i & l=i & (i=l,j),(k) & (n-1)(N-n) & \din & \dself & \eout & \Qout \\*
%
27 & j\neq i; j\in G_i & k\not\in G_i & l=j & (i, j=l), (k) & (n-1)(N-n) & \din & \din & \eout & \Qout \\*
%
28 & j\neq i; j\in G_i & k\not\in G_i & l\neq i, j; l\in G_i & (i, j, l), (k) & (n-1)(N-n)(n-2) & \din & \din & \eout & \Qout \\*
%
29 & j\neq i; j\in G_i & k\not\in G_i & l=k & (i, j), (k=l) & (n-1)(N-n) & \din & \dout & \eself & \Qout \\*
%
30 & j\neq i; j\in G_i & k\not\in G_i & l\neq k;l\in G_k & (i,j),(k,l) & (n-1)(N-n)(n-1) & \din & \dout & \ein & \Qout \\*
%
31 & j\neq i; j\in G_i & k\not\in G_i & l\not \in G_i, G_k & (i,j),(k),(l) & (n-1)(N-n)(N-2n) & \din & \dout & \eout & \Qout \\
%
%
%
%
32 & j\not\in G_i & k=i & l=i & (i=k=l),(j) & (N-n) & \dout & \dself & \eself & \Qout \\*
%
33 & j\not\in G_i & k=i & l\neq i; l\in G_i & (i=k, l), (j) & (N-n)(n-1) & \dout & \din & \ein & \Qout \\*
%
34 & j\not\in G_i & k=i & l=j & (i=k), (j=l) & (N-n) & \dout & \dout & \eout & \Qout \\*
%
35 & j\not\in G_i & k=i & l\neq j; l\in G_j & (i=k), (j,l) & (N-n)(n-1) & \dout & \dout & \eout & \Qout \\*
%
36 & j\not\in G_i & k=i & l\not \in G_i, G_j & (i=k), (j), (l) & (N-n)(N-2n) & \dout & \dout & \eout & \Qout \\
%
%
37 & j\not\in G_i & k\neq i; k\in G_i & l=i & (i=l, k), (j) & (N-n)(n-1) & \dout & \dself & \ein & \Qout \\*
%
38 & j\not\in G_i & k\neq i; k\in G_i & l=k & (i, k=l), (j) & (N-n)(n-1) & \dout & \din & \eself & \Qout \\*
%
39 & j\not\in G_i & k\neq i; k\in G_i & l\neq i,k; l\in G_i & (i, k, l), (j) & (N-n)(n-1)(n-2) & \dout & \din & \ein & \Qout \\*
%
40 & j\not\in G_i & k\neq i; k\in G_i & l=j & (i, k), (j=l) & (N-n)(n-1) & \dout & \dout & \eout & \Qout \\*
%
41 & j\not\in G_i & k\neq i; k\in G_i & l\neq j;l\in G_j & (i, k), (j,l) & (N-n)(n-1)(n-1) & \dout & \dout & \eout & \Qout \\*
%
42 & j\not\in G_i & k\neq i; k\in G_i & l\not \in G_i,G_j & (i, k), (j),(l) & (N-n)(n-1)(N-2n) & \dout & \dout & \eout & \Qout \\
%
%
43 & j\not\in G_i & k=j & l=i & (i=l), (j=k) & (N-n) & \dout & \dself & \eout & 1 \\*
%
44 & j\not\in G_i & k=j & l\neq i; l\in G_i & (i,l), (j=k) & (N-n)(n-1) & \dout & \din & \eout & 1\\*
%
45 & j\not\in G_i & k=j & l=j & (i), (j=k=l) & (N-n) & \dout & \dout & \eself & 1\\*
%
46 & j\not\in G_i & k=j & l\neq j; l\in G_j & (i), (j=k, l) & (N-n)(n-1) & \dout & \dout & \ein & 1\\*
%
47 & j\not\in G_i & k=j & l\not \in G_i, G_j & (i), (j=k), (l) & (N-n)(N-2n) & \dout & \dout & \eout & 1\\
%
%
48 & j\not\in G_i & k\neq j; k\in G_j & l=i & (i=l), (j, k) & (N-n)(n-1) & \dout & \dself & \eout & \Qin \\*
%
49 & j\not\in G_i & k\neq j; k\in G_j & l\neq i; l\in G_i & (i, l), (j, k) & (N-n)(n-1)(n-1) & \dout & \din & \eout & \Qin\\*
%
50 & j\not\in G_i & k\neq j; k\in G_j & l=j & (i), (j=l, k) & (N-n)(n-1) & \dout & \dout & \ein & \Qin\\*
%
51 & j\not\in G_i & k\neq j; k\in G_j & l=k & (i), (j, k=l) & (N-n)(n-1) & \dout & \dout & \eself & \Qin\\*
%
52 & j\not\in G_i & k\neq j; k\in G_j & l\neq j,k; l\in G_j & (i), (j, k, l) & (N-n)(n-1)(n-2) & \dout & \dout & \ein & \Qin\\*
%
53 & j\not\in G_i & k\neq j; k\in G_j & l\not \in G_i, G_j & (i), (j, k), (l) & (N-n)(n-1)(N-2n) & \dout & \dout & \eout & \Qin\\
%
%
54 & j\not\in G_i & k\not\in G_i, G_j & l=i & (i=l), (j), (k) & (N-n)(N-2n) & \dout & \dself & \eout & \Qout \\*
%
55 & j\not\in G_i & k\not\in G_i, G_j & l\neq i; l\in G_i& (i, l), (j), (k) & (N-n)(N-2n)(n-1) & \dout & \din & \eout & \Qout \\*
%
56 & j\not\in G_i & k\not\in G_i, G_j & l=j & (i), (j=l), (k) & (N-n)(N-2n) & \dout & \dout & \eout & \Qout \\*
%
57 & j\not\in G_i & k\not\in G_i, G_j & l\neq j; l\in G_j & (i), (j, l), (k) & (N-n)(N-2n)(n-1) & \dout & \dout & \eout & \Qout \\*
%
58 & j\not\in G_i & k\not\in G_i, G_j & l=k & (i), (j), (k=l) & (N-n)(N-2n) & \dout & \dout & \eself & \Qout \\*
%
59 & j\not\in G_i & k\not\in G_i, G_j & l\neq k; l\in G_k & (i), (j), (k, l) & (N-n)(N-2n)(n-1) & \dout & \dout & \ein & \Qout \\*
%
60 & j\not\in G_i & k\not\in G_i, G_j & l\not \in G_i, G_j, G_k & (i), (j), (k), (l) & (N-n)(N-2n)(N-3n) & \dout & \dout & \eout & \Qout \\
\end{longtable}
%\end{landscape}
}

\end{document}
