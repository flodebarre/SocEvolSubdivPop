
\makeatletter
\newlength\beamerleftmargin
\setlength\beamerleftmargin{\Gm@lmargin}
\makeatother

%%%%%%% STYLE %%%%%%%%%%%%%%%%%%%%%%%%%%%%%%%%%%%%%%%%%%%%
% FONTS
\usepackage[proportional]{sourcesanspro}
\usepackage[T1]{fontenc}

\usepackage{soul}   % Rayer

% Rayer a un certain moment seulement
\newcommand<>{\sta}[1]{
  \alt#2{\st{#1}}{{#1}}
}


\usepackage{hyperref}
\usepackage{multirow}

% TITLE
\setbeamerfont{frametitle}{size=\large}


% MARGINS
\newenvironment{changemargin}[2]{%
  \begin{list}{}{%
    \setlength{\topsep}{0pt}%
    \setlength{\leftmargin}{#1}%
    \setlength{\rightmargin}{#2}%
    \setlength{\listparindent}{\parindent}%
    \setlength{\itemindent}{\parindent}%
    \setlength{\parsep}{\parskip}%
  }%
  \item[]}{\end{list}} 

% FOOTNOTES
\renewcommand{\thefootnote}{}
\renewcommand{\footnoterule}{}

% MATHS
\usepackage{amsmath, amssymb, amsfonts}
%\usepackage[OMLmathrm,OMLmathbf]{isomath}

% LISTS
%\usepackage{enumitem}

% Enumerate style
\setbeamertemplate{enumerate items}[circle]
\setbeamercolor{item projected}{bg=maincol,fg=bgcol}

\setbeamertemplate{itemize subitem}{\color{maincol}$\blacktriangleright$}
\setbeamertemplate{itemize item}{\color{maincol}$\blacktriangleright$}

% Description
\setbeamercolor{description item}{ fg=maincol}

% COLORS
\usepackage{xcolor}
%\input{~/Dropbox/Presentations/Templates/tangocolors.sty}

\definecolor{bgcol}{rgb}{1,1,1} % Background color
\definecolor{fgcol}{rgb}{0,0,0} % Foreground color

\definecolor{maincol}{RGB}{0, 107, 137} % Bleu canard %006b8b
\definecolor{col1}{RGB}{125, 72, 150} % Purple %7d4896
\definecolor{col2}{RGB}{178, 208, 76} % Light green %b0d04c
\definecolor{col3}{RGB}{123, 191, 210} % Blue
\definecolor{col4}{HTML}{6C0013} % Opposite of light green
\definecolor{col12A}{RGB}{169, 152, 100}
\definecolor{col12B}{RGB}{42, 117, 125}

%\definecolor{maincol}{HTML}{7D1935} % Rubis
%\definecolor{col1}{HTML}{4A96AD} % Blue
%\definecolor{col2}{HTML}{575042} % Brown

\definecolor{cl1}{HTML}{234BCD}
\definecolor{cl2}{HTML}{5D5B8F}
\definecolor{cl3}{HTML}{976C52}
\definecolor{cl4}{HTML}{D17D15}

\definecolor{mgray}{gray}{0.55}
\definecolor{dgray}{gray}{0.3}
\definecolor{lgray}{gray}{0.75}
\definecolor{gray70}{gray}{0.70}

% COLORS AND THEMES
\usecolortheme[named=maincol]{structure} 
\setbeamercolor{alerted text}{fg=maincol} 
\setbeamercolor{frametitle}{fg=maincol}
\setbeamercolor{structure}{fg=fgcol, bg=bgcol}



% TITLE SECTION
\newcommand{\titlesec}[3]{\begin{center}
  \vspace{1.5cm}
                           \parbox{0.8\textwidth}{\textcolor{#3}{\begin{center} \Large #1 \\ \vspace{0.75cm} {\small \textcolor{mgray}{#2}} \end{center}}} \end{center}
			  }

% DESSINS
%\usepackage{pgf}
\usepackage{tikz}

\usetikzlibrary{patterns}
\usetikzlibrary{positioning,shadows,backgrounds,decorations.pathreplacing}
\usetikzlibrary{fit, calc}
\usetikzlibrary{shapes, shapes.callouts, decorations.text}
\usetikzlibrary{arrows}
%\usepackage{pgfplots}
\usepackage{rotating}

% GENERAL LAYOUT
\setbeamertemplate{navigation symbols}{} 
\setbeamertemplate{blocks}[rounded][shadow=false]

\setbeamercolor*{author in head/foot}{parent=palette quaternary}
\setbeamercolor*{title in head/foot}{parent=palette quaternary}
\setbeamercolor*{date in head/foot}{parent=palette quaternary}
\setbeamercolor*{section in head/foot}{fg=mgray}
\setbeamercolor*{subsection in head/foot}{parent=palette quaternary}

\def \hbarfoot {2.5ex}
\def \dbarfoot {1.75ex}
\defbeamertemplate*{footline}{infolines theme}
{
  \leavevmode%
  \hbox{%
  %
  \begin{beamercolorbox}[wd=.33\paperwidth, ht=\hbarfoot, dp=\dbarfoot, center]{section in head/foot}%
    \usebeamerfont{author in head/foot}
    \insertshortauthor % Uncomment to add author's name in the foot
      \end{beamercolorbox}
      %
  \begin{beamercolorbox}[wd=.33\paperwidth, ht=\hbarfoot, dp=\dbarfoot, center]{section in head/foot}%
    \usebeamerfont{title in head/foot}
% ESEB, August 2017
  \end{beamercolorbox}%
  %
  \begin{beamercolorbox}[wd=.33\paperwidth, ht=\hbarfoot, dp=\dbarfoot, right]{bg=bgcol, fg=fgcol}%
    \usebeamerfont{date in head/foot}
    \insertframenumber{} %/ \inserttotalframenumber %
	\hspace*{2ex} 
  \end{beamercolorbox}}%
  \vskip0pt%
}


\addtobeamertemplate{footline}{%
  %\leavevmode%
  \color{maincol!50!white}% to color the progressbar
%  \hspace*{-\beamer@leftmargin}%
%  \rule{\beamer@leftmargin}{2pt}%
  \rule{\dimexpr \insertframenumber\paperwidth/\inserttotalframenumber}{\dimexpr -1.25pt+\dbarfoot}
  % next 'empty' line is mandatory!

%  \vspace{0\baselineskip}
\vspace{\dimexpr -\hbarfoot-\dbarfoot}
  {}
}


% BLOCKS

%\addtobeamertemplate{block begin}{\pgfsetfillopacity{0.}}{\pgfsetfillopacity{1}}
%\addtobeamertemplate{block beamercolorbox begin}{\pgfsetfillopacity{0.65}}{\pgfsetfillopacity{1}}
%\setbeamercolor{block title}{fg=maincol, bg=red}%use=structure,fg=black,
%     bg=white!10}
\setbeamercolor{block title}{bg=white!10, fg=maincol}
%\setbeamercolor{block body}{%use=structure,fg=black,
%     bg=white!10}
\setbeamerfont{block title}{size=\normalsize}



\newcommand{\smallblock}[3]{
\begin{minipage}[c]{#1}
  \begin{block}{#2}
   #3
  \end{block}
\end{minipage}
}





% ROTATE TEXT FOR CREDITS
\usepackage{graphics}
\definecolor{lgray}{gray}{0.75}
\newcommand{\piccredit}[1]{\hspace{0.1em}\rotatebox{90}{\tiny \textcolor{gray70}{(c) #1}}}
\newcommand{\picc}[1]{ \rotatebox{90}{\tiny \textcolor{lgray}{#1}}}

% STROKE
%\usepackage{ulem}

% TABLES
\usepackage{array}

% ANIMATIONS - MOVIES
\usepackage{multimedia}


\newcommand{\refpaper}[3]{
\begin{flushright}
\textcolor{gray50}{\tiny #1 (#2), \textit{#3}}
\end{flushright}
}

%% Maths

%\newcommand{\mat}[1]{\mathrm{\mathbf{#1}}}
\newcommand{\mat}[1]{\mathbf{#1}}
\DeclareMathOperator{\Tr}{Tr}
\newcommand{\Trace}[1]{\Tr \left( #1 \right)}

%% Command for Raising pictures
\newcommand{\raisepic}[1]{\raisebox{-\height}{#1}}
\newcommand{\midpic}[2]{\raisebox{-#2\height}{#1}}

% Checkmarks
\usepackage{pifont}
\newcommand{\cmark}{\ding{51}}%
\newcommand{\xmark}{\ding{55}}%



% REFERENCES
%% References
\newcommand{\pprref}[2]{\textcolor{mgray}{#1 (#2)}}
%\usepackage{natbib}
% Remove the icon before each item
\setbeamertemplate{bibliography item}{}

% Only number References slides if they are more than 1
\setbeamertemplate{frametitle continuation}[from second]

%remove line breaks
\setbeamertemplate{bibliography entry title}{}
\setbeamertemplate{bibliography entry location}{}
\setbeamertemplate{bibliography entry note}{}

%\setbeamercolor*{bibliography entry title}{fg=fgcol}
\setbeamercolor*{bibliography entry author}{fg=maincol}
%\setbeamercolor*{bibliography entry note}{fg=fgcol}
%\setbeamercolor*{bibliography entry location}{fg=fgcol}


% In text?
\newcommand{\refstyle}[1]{\scriptsize  \textcolor{dgray}{#1}}

% As footnote
\newcommand{\theref}[1]{{\footnotetext{\begin{flushright} \refstyle{#1} \end{flushright} } }}

% NOTES
%\setbeamertemplate{note page}[plain]
\setbeamerfont{note page}{size=\scriptsize}

% APPENDIX
% Change numbering for the appendix
\newcommand{\backupbegin}{
   \newcounter{framenumberappendix}
   \setcounter{framenumberappendix}{\value{framenumber}}
}
\newcommand{\backupend}{
   \addtocounter{framenumberappendix}{-\value{framenumber}}
   \addtocounter{framenumber}{\value{framenumberappendix}} 
}

% HYPERLINKS
\setbeamercolor{button}{bg=maincol,fg=bgcol}

% DESCRIPTIONS
\defbeamertemplate{description item}{align left}{\insertdescriptionitem\hfill}


\newcommand{\btVFill}{\vskip0pt plus 1filll}
\newcommand{\EE}{\mathbb{E}}
\newcommand{\PP}{\mathbb{P}}

%%%%%%%%%%%%%%%%%%%%%%%%%%%%%%%%%%%%%%%%%%%%%%%%%%%%%%

%\colorlet{colA}{ta2chameleon}
%\colorlet{colB}{ta2chocolate}

% Colors for the legends
\definecolor{pinka}{HTML}{DA8DAC}
\definecolor{pinkb}{HTML}{ac0045}
\definecolor{pinkc}{HTML}{4c001e}
\definecolor{pinkd}{HTML}{ecc6d5}


%\newcommand{\sderivv}[3]{\left.\frac{\partial #1}{\partial #2}\right|_{#3 =0}}
\newcommand{\sderivv}[3]{\left. \frac{\partial #1}{\partial #2} \right|_{\delta =0}}

\newcommand{\ssum}[2]{{\textstyle \sum\limits_{#1}^{#2}}}

\newcommand{\bb}{\mathsf{b}}
\newcommand{\cc}{\mathsf{c}}
\newcommand{\dd}{\mathsf{d}}

\newcommand{\demesize}{n}
\newcommand{\nbdemes}{N_d}
\newcommand{\Qin}{Q_{\textsf{in}}}
\newcommand{\Qout}{Q_{\textsf{out}}}
\newcommand{\selstr}{\delta}