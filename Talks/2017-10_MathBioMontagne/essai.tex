\documentclass[aspectratio=169]{beamer}

\input{styleMBM.tex}

\begin{document}
\def \wpic {0.3\textwidth}


\begin{frame}{Take-Home Messages}


\begin{center}
\begin{minipage}{0.9\textwidth}

\begin{itemize}
\item<+-> Under weak selection, it is
possible to compute the expected frequency of social
individuals, for any life-cycle, any regular population structure,  any mutation probability. \hspace{\stretch{1}}{\footnotesize \textcolor{mgray}{(D., 2017, JTB)}}

\item<+-> $\mathbb{E}[\overline{X}]>\nu \Leftrightarrow \mathcal{B}\, R > \mathcal{C}$.

\item<+-> In subdivided populations, $\mathbb{E}[\overline{X}]$ can increase with the emigration probability $m$ when strategy transmission is imperfect ($\mu > 0$).  \hspace{\stretch{1}}{\footnotesize \textcolor{mgray}{(D., \textit{in review.})}}


\item<+-> This result seems to hold under stronger selection \\and in heterogeneous populations.
\end{itemize}
\end{minipage}
\end{center}


\uncover<+->{


\btVFill
\hrule
\begin{block}{Funding \& Thanks}
\begin{center}
\vspace{-0.25cm}

\begin{tikzpicture}
\node[](anr){\includegraphics[height = 1cm]{/home/florencedebarre/Dropbox/Presentations/GlobalPics/logo_anr.pdf}};
\node[below = 0cm of anr, font = \small](anrn){ANR-14-ACHN-0003-01};

\node[right = of anrn.south east, align = center, anchor = south west](Ch){A. Lambert \\ for organizing the workshop};

\node[right = of Ch, color = maincol, text width = 4cm, font = \large, align = center]{and thank you for your attention!};

\end{tikzpicture}
\end{center}

\end{block}
}
\end{frame}

\begin{frame}

\end{frame}
\end{document}
